\documentclass{report}
\setcounter{chapter}{1}
\input{../latex_template/preamble}
\input{../latex_template/macros}
\input{../latex_template/letterfonts}
\graphicspath{ {./images/} }


\title{$SO(3)$ and $so(3)$ \cite{slides}\\
    \large Lie group and Lie algebra for 3D rotation matrices
}
\date{2022 October}
\author{Davide Malvezzi}

\begin{document}
\maketitle

\section{Skew-symmetric matrices}
\dfn{Skew-symmetric matrix}{
    A matrix $A \in \mathbb{R}^{n \times n}$ is called skew-symmetric or anti-symmetric if 
    \begin{align}
        A^T = -A
    \end{align}
}
\noindent Given a vector $w \in \mathbb{R}^3$ the hat operator give the corresponding skew-symmetric matrix
\begin{align}
    \hat{w} = \left[ w \right]_{\times} = 
    \left(
    \begin{matrix}
        &0 &-w_3 &-w_2 \\   
        &w_3 &0 &-w_1 \\
        &-w_2 &w_1 &0
    \end{matrix}
    \right)
\end{align}
In particular
\begin{align}
    \hat{w}u = \left[ w \right]_{\times} u = w {\times} u
\end{align}

\section{$SO(3)$ and $so(3)$}
\dfn{Lie group}{
    A Lie group (or infinitesimal group) is a smooth manifold that is also a group, such that the group operations multiplication and inversion are smooth maps.
}
\noindent We define as the special orthogonal group $SO(3)$ the Lie group of all 3D rotation matrices
\begin{align}
    SO(3) = \left\{  R \in \mathbb{R}^{3x3} \ | \ R^TR = I, \det(R)=1 \right\}
\end{align} 
Let's now consider a family of rotations $R(t)$ with $R(0)=I$ which continuously transform a point from its original location
\begin{align}
    X(t) = R(t)X_0 \quad R(t) \in SO(3)
\end{align}
since $R(t)R(t)^T=I$ then
\begin{align}
    \frac{d}{dt}(RR^T) = \dot{R}R^T + R\dot{R}^T = 0 \quad \rightarrow \quad \dot{R}R^T = -(\dot{R}R^T)^T 
\end{align}
It follow that $\dot{R}R^T$ is a skew-symmetric matrix, so
\begin{align}
    \dot{R}R^T = \hat{w}(t) \quad \rightarrow \quad \dot{R}= \hat{w}(t)R
\end{align}
Since $R(0) = I$ then $\dot{R}(0) = \hat{w}(0)$, therefore from the first order Taylor expansion
\begin{align}
    R(dt) = R(0) + dR = I + \hat{w}(0)dt
\end{align}
We define the space $so(3)$ as
\begin{align}
    so(3) = \left\{\hat{w} \ | \ w \in \mathbb{R}^3\right\}
\end{align}
The $so(3)$ is the Lie algebra of the Lie group $SO(3)$.
\dfn{Lie algebra}{
    A Lie group gives rise to a Lie algebra, which is its tangent space at the identity. 
}

\subsection{Exponential map}
Given the differential equation system
\begin{align}
    \begin{cases}
        \dot{R}(t)= \hat{w}R(t)\\
        R(0) = I
    \end{cases}
\end{align}
the solution is
\begin{align}
    R(t) = e^{\hat{w}t} = \sum_{n=0}^{\infty} \frac{(\hat{w}t)^n}{n!} = I + \hat{w}t + \frac{(\hat{w}t)^2}{2} + \dots
\end{align}
This matrix exponential defines a map from the Lie algebra to the Lie group
\begin{align}
    \text{exp : } so(3) \rightarrow SO(3)
\end{align}
which is a rotation around the axis $w \in \mathbb{R}^3$ by an angle of $t$ if $|w| = 1$.

\subsection{Logarithm map}
The logarithm map is the inverse of the exponential map
\begin{align}
    \text{log : } SO(3) \rightarrow so(3)
\end{align}
Given $R \in SO(3), R \neq I$
\begin{align}
    \lvert w \rvert = \cos^{-1}\left( \frac{\text{tr}(R) - 1}{2} \right) \\
    \frac{w}{\lvert w \rvert} = \frac{1}{2\sin(\lvert w \rvert)} 
    \left(
    \begin{matrix}
        r_{32} - r_{23}\\
        r_{13} - r_{31}\\
        r_{21} - r_{12}
    \end{matrix}
    \right)
\end{align}

\nt{}{
    The above statement says any orthogonal transformation $R \in SO(3)$ 
    can be realized by rotating by an angle $|w|$ around an axis $\frac{w}{|w|}$ 
    as defined above.
}
\noindent The above representation is not unique because increasing the angle by multiples of 
$2\pi$ will give the same rotation.

\subsection{Rodrigues' Formula}
Given $w \in \mathbb{R}^3$, the exponential map can be computed in closed form as
\begin{align}
    e^{\hat{w}} = I + \frac{\hat{w}}{|w|}\sin(|w|) + \frac{\hat{w}^2}{|w|^2}(1-\cos(|w|))
\end{align}

\section{$SE(3)$ and $se(3)$}
In the same way it is possible to define a Lie group and a Lie algebra for the rigid-body transformation.
\dfn{Special euclidian group $SE(3)$}{
    \begin{align}
        SE(3) = \left\{  
            g = \left(
            \begin{matrix}
                &R &T&  \\
                &0 &1& 
            \end{matrix} 
            \right) \ | \ R \in SO(3), T \in \mathbb{R}^3
        \right\} \subset \mathbb{R}^{4 \times 4}
    \end{align}
}
Given $g(t) = \left(
    \begin{matrix}
        &R(t) &T(t)&  \\
        &0 &1& 
    \end{matrix} 
    \right)$ we have
\begin{align}
    \dot{g}(t)g^{-1}(t) = \left(
        \begin{matrix}
            &\dot{R}R^T &\dot{T} - \dot{R}R^TT &  \\
            &0 &1& 
        \end{matrix} 
        \right)
\end{align}
As in the case of $SO(3)$, the matrix  $\dot{R}R^T$ corresponds 
to some skew-symmetric matrix $\hat{w} \in so(3)$
\begin{align}
    \dot{g}(t)g^{-1}(t) = \left(
        \begin{matrix}
            &\hat{w}(t) & v(t) &  \\
            &0 &1& 
        \end{matrix} 
        \right) = \hat{\xi}(t)
\end{align}
So 
\begin{align}
    \dot{g}= \dot{g}g^{-1}g = \hat{\xi}g
\end{align}
therefore $\hat{\xi}$ is the tangent space of $g(t)$ and it is called a twist.
\dfn{$se(3)$}{
    The Lie algebra of $SE(3)$ is
    \begin{align}
        se(3) = \left\{ \hat{\xi} = 
            \left(
            \begin{matrix}
                &\hat{w} &v \\
                &0 &0
            \end{matrix}
            \right) \ | \ \hat{w} \in so(3), v \in \mathbb{R}^3
        \right\}
    \end{align}
}

\subsection{Exponential map}
For $\hat{w} = 0$, $e^{\hat{\xi} t} = \left( 
\begin{matrix}
&I &v& \\
&0 &1&    
\end{matrix}
\right)$, otherwise
\begin{align}
    e^{\hat{\xi}} = \left( 
        \begin{matrix}
        &e^{\hat{w}} &\frac{(I-e^{\hat{w}})\hat{w}v+ww^Tv}{|w|^2}& \\
        &0 &1&    
        \end{matrix}
        \right)
\end{align}

\subsection{Logarithm map}
Given $g = (R, T)$, we know that there exists $w \in \mathbb{R}^3$ with $e^{\hat{w}} = R$. \\
If $|w| \neq 0$, the exponential form of $g$ introduced above shows that we merely need to solve the equation
\begin{align}
    \frac{(I-e^{\hat{w}})\hat{w}v+ww^Tv}{|w|^2} = T
\end{align}
for the vector $v \in \mathbb{R}^3$.\\
Just as in the case of $SO(3)$, this representation is generally not unique, because there exist many
twists which represent the same rigid-body motion.

\bibliographystyle{plain}
\bibliography{refs}

\end{document}
